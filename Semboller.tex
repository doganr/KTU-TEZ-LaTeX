\newpage
\pagestyle{plain}
\addcontentsline{toc}{chapter}{\hspace{-0.75cm}SEMBOLLER D\.{I}Z\.{I}N\.{I}}
\null\vspace{-1.5cm}%
\begin{center}
{\bf SEMBOLLER D\.{I}Z\.{I}N\.{I}}
\end{center}
\vspace{0.5cm}

\begin{longtable}{p{3.5cm}p{0.2cm}p{9.5cm}}
    $ACC $ & : &  Doğruluk (Accuracy) \\
    $AdaGrad $ & : &  Adaptif Gradyan Algoritması (Adaptive Gradient) \\
    $Adam $ & : &  Adaptif Moment Tahmini algoritması (Adaptive Moment Estimation) \\
    $AUC $ & : &  ROC Eğrisinin Alt Alanı (Area under the ROC Curve) \\
    $BCE $ & : &  İkili Çapraz Entropi Kayıp (Binary Cross Entropy Loss) \\
    $BGD $ & : &  Paket gradyan azalma (Batch Gradient Descent) \\
    $BN $ & : &  Paket Normalizasyonu (Batch Normalizasyonu) \\
    $BT $ & : &  Bilgisayarlı Tomografi \\
    $C2FNAS $ & : &  Kabadan İnceye Sinir Mimari Araması (Coarse-to-Fine Neural Architecture Search) \\
    $CAKES $ & : &  Kanal Otomatik Çekirdek Küçültmesini (Channel wise Automatic KErnel Shrinking) \\
    $CCE $ & : &  Kategorik Çapraz Entropi Kaybı (Categorical Cross Entropy Loss) \\
    $CNN $ & : &  Konvolüsyonel Sinir Ağı (Convolutional Neural Network) \\
    $CRNN $ & : &  Bağlamsal Tekrarlayan Sinir Ağları (Contextual Recurrent Neural Networks) \\
    $DN $ & : &  Doğru Negatif \\
    $DNN $ & : &  Derin Sinir Ağları (Deep Neural Network) \\
    $DÖ $ & : &  Derin Öğrenme \\
    $DP $ & : &  Doğru Pozitif \\
    $DSC $ & : &  Dice Benzerlik Katsayısı (Dice Similarity Coefficient) \\
    $ELU $ & : &  Eksponansiyel Doğrusal Birim (Exponential Linear Unit) \\
    $ERCP $ & : &  Endoskopik Retrograd Kolanjiopankreatografi \\
    $EUS $ & : &  Endoskopik Ultrason \\
    $Faster \; R-CNN $ & : &  Hızlandırılmış Bölge Esaslı Konvolüsyonel Sinir Ağları (Faster Region     Proposal Convolutional Neural Networks) \\
    $FC $ & : &  Tam Bağlantılı (Fully Connected) \\
    $FCN $ & : &  Tam Konvolüsyonel Ağlar (Fully Convolutional Networks) \\
    $FL $ & : &  Odak Kaybı  (Focal Loss ) \\
    $FNN $ & : &  İleri Beslemeli Sinir Ağları (Feed Forward Neural Network) \\
    $FPN $ & : &  Özellik Piramit Ağı (Feature Pyramid Network) \\
    $GPU $ & : &  Grafik İşlemci Ünitesi (Graphics Processing Unit) \\
    $HL $ & : &  Menteşe Kayıp (Hinge Loss) \\
    $ILSVRC $ & : &  ImageNet Büyük Ölçekli Görsel Tanıma Yarışması (ImageNet Large Scale Visual     Recognition Challenge) \\
    $JI $ & : &  Jaccard İndeksi (Jaccard Index) \\
    $LSTM $ & : &  Uzun-Kısa Vadeli Bellek (Long Short Term Memory) \\
    $MAE $ & : &  Ortalama Mutlak Hata (Mean Absolute Error) \\
    $Mask \; R-CNN $ & : &  Maskelenmiş Bölge Esaslı Konvolüsyonel Sinir Ağı (Masked Region Proposal     Convolutional Neural Network) \\
    $MBE $ & : &  Ortalama Önyargı Hatası (Mean Bias Error) \\
    $MGN $ & : &  Maske Üretim Ağı (Mask Generation Network) \\
    $MICCAI $ & : &  Tıbbi Görüntü Hesaplama ve Bilgisayar Destekli Müdahaleler Konferansı (Medical     Image Computing and Computer Aided Interventions Conference) \\
    $MLP $ & : &  Çok Katmanlı Algılayıcılar (Multi Layer Perceptron) \\
    $MR $ & : &  Manyetik Rezonans \\
    $MSD $ & : &  Tıbbi Segmentasyon Dekatlon Veri Seti (Medical Segmentation Decathlon) \\
    $MSE $ & : &  Ortalama Karesel Hata (Mean Square Error) \\
    $NAS $ & : &  Yapay Mimari Arama (Neural Architecture Search) \\
    $NIH $ & : &  Ulusal Sağlık Enstitüleri Pankreas BT Veri Seti (National Institutes of Health     Clinical Center Pancreas CT) \\
    $nnUNet $ & : &  No-New-Net \\
    $PAN $ & : &  Projektif Çekişmeli Ağlar (Projective Adversarial Networks) \\
    $PRE $ & : &  Kesinlik (Precision) \\
    $PReLU $ & : &  Parametrik ReLU (Parametric ReLU) \\
    $R-CNN $ & : &  Bölge Tabanlı Evrişimli Sinir Ağı (Region-based Convolutional Neural Network) \\
    $REC $ & : &  Duyarlılık (Sensivity, Recall) \\
    $ReLU $ & : &  Doğrultulmuş Doğrusal Birim (Rectified Linear Unit) \\
    $ResNet $ & : &  Derin Artık Ağlar (Deep Residual Networks) \\
    $RMSProp $ & : &  Ortalama Karekök Yayılım algoritması (Root Mean Square Propagation) \\
    $RNN $ & : &  Özyineli Sinir Ağları (Recurrent Neural Networks) \\
    $ROC $ & : &  Alıcı Çalışma Karakteristikleri (Receiver Operating Characteristics) \\
    $RPN $ & : &  Aday Bölge Çıkarım Ağı (Region Proposal Network) \\
    $SEER $ & : &  The Surveillance, Epidemiology, and End Results \\
    $SGD $ & : &  Stokastik Gradyan Azalma (Stochastic Gradient Descent) \\
    $SPE $ & : &  Özgüllük (Specificity) \\
    $SSD $ & : &  Tek Atış Dedektörü (Single-shot Detector) \\
    $UMCT $ & : &  Belirsizliğe Duyarlı Çoklu Görüş Ortak Eğitimi (Uncertainty-aware Multi-view     Co-training) \\
    $VGG $ & : &  Çok Derin Konvolüsyonel Ağlar (Very Deep Convolutional Networks) \\
    $YOLO $ & : &  You Only Look Once \\
    $YN $ & : &  Yanlış Negatif \\
    $YP $ & : &  Yanlış Pozitif \\        
\end{longtable}

% Bu örnek bir semboller belgesidir isterseniz 
%\begin{description}
%content...
%\end{description} 
% komutları kullanarak da simglere dizini oluşturabilirsiniz. 