\documentclass[hidelinks]{ktutez}
\usepackage[top=40mm, bottom=25mm, left=30mm, right=25mm]{geometry}
\usepackage[T1]{fontenc}
\usepackage[utf8]{inputenc}
\usepackage{lmodern}
\usepackage{ragged2e}
\usepackage[dotinlabels]{titletoc}
\usepackage[titletoc]{appendix}
\usepackage{tocloft}
\usepackage{titlesec}
\usepackage{rotating}
\usepackage{epstopdf}
\usepackage[table,usenames,dvipsnames]{xcolor}
\usepackage{multirow}
\usepackage{longtable}
\usepackage{tabularx}
%\usepackage{tabu}
\usepackage{array}
\usepackage{lscape}
\usepackage{indentfirst}
\usepackage{afterpage}
%\usepackage[backend=biber,style=numeric,sorting=none]{biblatex}
\usepackage[bookmarks=false]{hyperref}
\usepackage{etoolbox}
\usepackage{booktabs}
\usepackage{url}
\usepackage[none]{hyphenat}
\usepackage{fp}
\usepackage[font=normalsize]{caption}
\usepackage{tikz}
\usetikzlibrary{shapes,arrows,shadows}
\usepackage[fleqn]{amsmath}
%\usepackage{bm,times}
\usepackage{bm,times}
%\usepackage{fontspec}
%\setmainfont{Times New Roman}
\usepackage{verbatim}
\usepackage{wrapfig}
\usepackage{lipsum}
\usepackage{array}
\usepackage{lastpage}
\usepackage{soul}
\usepackage{blindtext}
\usepackage{mathptmx}
\usepackage{hepnicenames}
%\usepackage{enumerate}
\usepackage{microtype}
\usetikzlibrary{shapes.geometric, arrows}
\usepackage[justification=justified,singlelinecheck=false, font=singlespacing]{caption}
\usepackage{setspace}
\usepackage{placeins}
\usepackage{xcolor}
\usepackage{fancyhdr}
\usepackage[acronym,toc]{glossaries}
\usepackage{color, colortbl}
\definecolor{Gray}{gray}{0.9}

%referanslarda makalelerin altini cizdirmek icin trick
\usepackage {ulem}
%referanslarda url lerin fontunu duzeltmek icin trick
\urlstyle{same}


%\usepackage[noadjust]{cite}
%\renewcommand{\citedash}{-}

\usepackage{etoolbox}%chapter baslangic sayfası numarayı gösterme
\patchcmd{\chapter}{plain}{empty}{}{}
\patchcmd{\part}{plain}{empty}{}{}
%\usepackage{caption}
\usepackage{mwe} % just for dummy images
% % % % % % % % % % % % % % % % % % % % % % % % % % % % % % % % % % % % % % % % % % % % % % % % % % % % % % % % % % % % %
% % % % % % % % % % % % % % % % % % % % % % % % % % % % % % % % % % % % % % % % % % % % % % % % % % % % % % % % % % % % %
%%% For tables
%\usepackage{multirow}
% Longtable lets you have tables that span multiple pages.
%\usepackage{longtable}
%for equations
%\usepackage{amsmath}
\usepackage{amssymb}
%\usepackage{algorithmic}
\usepackage{algorithm}
%\usepackage{algorithmicx}
\usepackage[noend]{algpseudocode}
%\usepackage[]{algorithm2e}
% Booktabs produces far nicer tables than the standard LaTeX tables.
%   see: http://en.wikibooks.org/wiki/LaTeX/Tables
%\usepackage{booktabs}
%\usepackage{afterpage}
%\usepackage[toc]{glossaries}
%set parameters for longtable:
% default caption width is 4in for longtable, but wider for normal tables
%\setlength{\LTcapwidth}{\textwidth}
%\usepackage[demo]{graphicx}
%\usepackage[position=t,singlelinecheck=off]{subfig}
%\usepackage{caption}
%%%%%%%%%%YATAY TABLO
%\usepackage{graphicx,rotating,booktabs}
\usepackage{graphicx}
\usepackage{subcaption}
%\usepackage[verbose]{placeins}
%\usepackage[ngerman]{babel}
%\usepackage{blindtext}
%tablolar ve figurler listesinde cite order i durdurmak icin
\usepackage{notoccite}
\usepackage{adjustbox}
\usepackage{pbox}

%%%%%%%%%%YATAY TABLO
\captionsetup[table]{position=t}

%% Bu satırdan itibaren bazı ayar komutları başlamaktadır, eğer fazla bilginiz yoksa bu satırlarda değişiklik yapmayın.
\captionsetup{labelsep=period}
\newcommand{\mx}[1]{\mathbf{\bm{#1}}} % Matrix command
\newcommand{\vc}[1]{\mathbf{\bm{#1}}} % Vector command

\newsavebox\mytabularbox
\newcount\figwidthc
\newcount\textwidthc

\renewcommand{\cftdotsep}{0.1}%%icindekilerde noktalar arasi boosluk \titlerule ile ayni
\titlecontents{chapter}[0.9em]{\addvspace{0em}}{\contentslabel{1.7em}\hspace*{2.8em}}{.}{\titlerule*[0.2pc]{.}\contentspage}

\titleformat{\chapter}{\normalfont\fontsize{12}{12}\bfseries}{\thechapter . \hspace{0.2em}}{0em}{}

\titleformat{\section}{\normalfont\fontsize{12}{12}\bfseries}{\thesection . \hspace{0.2em}}{0em}{}

\titleformat{\subsection}{\normalfont\fontsize{12}{12}\bfseries}{\thesubsection . \hspace{0.2em}}{0em}{}
  
\titleformat{\subsubsection}{\normalfont\fontsize{12}{12}\bfseries}{\thesubsubsection . \hspace{0.2em}}{0em}{}  
 
\titleformat{\paragraph}{\normalfont\fontsize{12}{12}\bfseries}{\theparagraph . \hspace{0.2em}}{0em}{}  

%% içindekiler numara başlık arasındaki boşluk
\cftsetindents{chapter}{-0.9em}{4.7em} 
\cftsetindents{section}{-0.9em}{4.7em}
\cftsetindents{subsection}{-0.9em}{4.7em}
\cftsetindents{subsubsection}{-0.9em}{4.7em}
\cftsetindents{paragraph}{-0.9em}{4.7em}
  
\pagenumbering{Roman}
\setcounter{page}{3}

%%tez meta bilgisini cagiriyoruz

% % % % % % % % % % % % % % % % % % % % % % % % % % % % % % % % % % % % % %
% % Bu satırdan sonra teziniz ile ilgili girmeniz gereken bilgiler bulunmaktadır. Kıvrık parantezler arasına size uygun bilgileri girebilirsiniz.
%%**************************************************************************
%\renewcommand{\teztipibool}{0}
%\thesistype{MS. Thesis}
%\teztipi{YÜKSEK LİSANS TEZİ}
%\teztipikucuk{Yüksek Lisans Tezi}
%%**************************************************************************
\renewcommand{\teztipibool}{1}
\thesistype{PhD. Thesis}
\teztipi{DOKTORA TEZİ}
\teztipikucuk{Doktora Tezi}
%**************************************************************************
\keywords{
	\singlespacing
	Computer tomography, Pancreas segmentation, Pancreatic tumor segmentation, Deep learning, Mask R-CNN, 3B U-Net
	%Serous effusion, Cytopathology, Machine learning, Computer aided diagnosis, Stain normalization, Nuclei detection, %Nuclei segmentation, Cell classification, Deep learning, Convolutional neural networks.
} % Anahtar kelimelerin İngilizce'sini aralarına virgül koyarak yazın

\anahtarsoz{
	\singlespacing
	Bilgisayarlı tomografi, Pankreas segmentasyonu, Pankreas tümörü segmentasyonu, Derin öğrenme, Mask R-CNN, 3B U-Net
}% Anahtar kelimelerin Türkçe'sini aralarına virgül koyarak yazın

\title{INVESTIGATION OF DIFFERENT DEEP LEARNING TECHNIQUES IN PANCREAS CANCER TISSUES SEGMENTATION} % Tez başlığınızın İngilizce'sini BÜYÜK harflerle yazın
\titlesmall{Investigation of Different Deep Learning Techniques in Pancreas Cancer Tissues Segmentation}% Tez başlığınızın İngilizce'sini küçük harflerle yazın
\baslik{PANKREAS KANSER DOKULARININ SEGMENTASYONUNDA FARKLI DERİN ÖĞRENME TEKNİKLERİNİN PERFORMANSLARININ İNCELENMESİ} % Tez başlığınızın Türkçe'sini BÜYÜK harflerle yazın
\baslikkucuk{Pankreas Kanser Dokularının Bölütlenmesinde Farklı Derin Öğrenme Tekniklerinin Performanslarının İncelenmesi}% Tez başlığınızın Türkçe'sini küçük harflerle yazın
\yazar{RAMAZAN ÖZGÜR DOĞAN}    % Tez yazarının adi ve soyadı BÜYÜK harflerle yazilmali
\yazarkucuk{Ramazan Özgür DOĞAN}    % Tez yazarının adı normal soyadı BÜYÜK harflerle yazilmali
\yunvan{Bil. Yük. Müh.}
\yorcid{0000-0001-6415-5755}
% % % % % % % % % % % % % % % % % % % % % % % % % % % % % % % % % % % % % % % % % % % % % % % % % % % % % % % % % % % % %
% % % % % % % % % % % % % % % % % % % % % % % % % % % % % % % % % % % % % % % % % % % % % % % % % % % % % % % % % % % % %
% %Aşağıdaki üniversite adı ve enstitü adında genel bir değişiklik yapmanıza gerek yoktur ama ihtiyaça halınde nasıl değişiklik yapabileceğiniz her satırn yanında yazmaktadır.
\universite{KARADENİZ TEKNİK ÜNİVERSİTESİ} %Üniveritenin adının Türkçe'sini BÜYÜK harflerle yazın
\universitekucuk{Karadeniz Teknik Üniversitesi}%Üniveritenin adının Türkçe'sini yazın
\university{KARADENIZ TECHNICAL UNIVERSITY}%Üniveritenin adının İngilizce'sini BÜYÜK harflerle yazın
\universitysmall{Karadeniz Technical University}%Üniveritenin adının İngilizce'sini yazın
\misafiruniversite{disaridan gelen hocanın üniversitesi} %Dışarıdan gelen hocanın bağlı bulunduğu üniversiteyi buraya yazın, eğer başka üniversiteden gelen hoca yoksa buraya da kendi üniversitenizin adını yazın.

\enstitu{FEN BİLİMLERİ ENSTİTÜSÜ} %Enstitünün adının Türkçe'sini BÜYÜK harflerle yazın
\enstitukucuk{Fen Bilimleri Enstitüsü} %Enstitünün adının Türkçe'sini yazın
\institute{THE GRADUATE SCHOOL OF NATURAL AND APPLIED SCIENCES}%Enstitünün adının İngilizce'sini BÜYÜK harflerle yazın
\institutesmall{The Graduate School of Natural and Applied Sciences}%Enstitünün adının İngilizce'sini yazın
\ounvan{DOKTOR (BİLGİSAYAR MÜHENDİSLİĞİ)}% alacağı ünvan

% % % % % % % % % % % % % % % % % % % % % % % % % % % % % % % % % % % % % % % % % % % % % % % % % % % % % % % % % % % % %
% % % % % % % % % % % % % % % % % % % % % % % % % % % % % % % % % % % % % % % % % % % % % % % % % % % % % % % % % % % % %

\bolum{BİLGİSAYAR MÜHENDİSLİĞİ} % Bölümünüzün adının Türkçe'sini BÜYÜK harflerle yazın
\bolumkucuk{Bilgisayar Mühendisliği} % Bölümünüzün adının Türkçe'sini yazın
\misafirbolum{Bilgisayar Mühendisliği} %Farklı anabilin dalından gelen hoca için burayı değiştirebilirsiniz.
\dept{COMPUTER ENGINEERING GRADUATE PROGRAM}% Bölümünüzün adının İngilizce'sini BÜYÜK harflerle yazın
\deptsmall{Computer Engineering Graduate Program}% Bölümünüzün adının İngilizce'sini  yazın

\supervisor{Prof. Dr. Temel KAYIKÇIOĞLU} %Danışmanınızın ünvanını İngilizce olarak yazın
\danisman{Prof. Dr. Temel KAYIKÇIOĞLU}%Danışmanınızın ünvanını Türkçe olarak yazın
\dorcid{0000-0002-6787-2415}
\juriuyesi{Prof. Dr. Onur OSMAN} %Diğer jüri üyesini buraya yazınız.
\secondreader{Prof. Dr. Temel KAYIKÇIOĞLU} %Diğer jüri üyesini buraya yazınız.
\thirdreader{Prof. Dr. Cemal KÖSE} %Diğer jüri üyesini buraya yazınız.
\forthreader{Prof. Dr. Ali GANGAL} %Diğer jüri üyesini buraya yazınız.
\fifthreader{Prof. Dr. Alper BAŞTÜRK} %Diğer jüri üyesini buraya yazınız.
\enstitumuduru{Prof. Dr. Asim KADIOĞLU} %Enstitü müdürünün adını yazınız.


\copyrightyear{2021}
\submitdate{xxxxx}  % Tezin verilme tarihi İngilizce BÜYÜK harflerle yazilmali
\submitdatesmall{xxxxx} % Tezin verilme tarihi İngilizce harflerle yazilmali
\tarih{ARALIK 2021}   % Tezin verilme tarihi Türkçe BÜYÜK harflerle yazilmali
\tarihkucuk{Aralık/2021}     % Tezin verilme tarihi Türkçe harflerle yazilmali
\onaytarihi{02/12/2021} % Tezinizin onay tarihini buraya yazınız.
\onaysayisi{1929} % Tezinizin onay sayısını buraya yazınız.
\evertarihi{23/11/2021} % Tezinizin enstitüye verildiği tarihi bura yazınız.
\evertarihii{23/11/2021} % Tezinizin enstitüye verildiği tarihi bura yazınız.
\savunmatarihi{16/12/2021} % Tezinizin savunma tarihini bura yazınız.
\onaysekli{oy birliği} % Eğer oy çokluğu ile teziniz kabul edildi ise "oy birliği" yerine "oy çokluğu" yazınız.


% % % % % % % % % % % % % % % % % % % % % % % % % % % % % % % % % % % % % % % % % % % % % % % % % % % % % % % % % % % % %
% % % % % % % % % % % % % % % % % % % % % % % % % % % % % % % % % % % % % % % % % % % % % % % % % % % % % % % % % % % % %


%%semboller dizini icin
%\include{Semboller2}

\newtheorem{thm}{Theorem}
\newtheorem{preexam}{Example}
\newenvironment{exam} {\begin{preexam}\rm}{\end{preexam}}
\newtheorem{lem}{Lemma}
\newtheorem{proprty}{Property}
\newtheorem{cor}{Corollary}
\newtheorem{defn}{Definition}
\newcommand{\pe}{\preceq}
\newcommand{\po}{\prec}

\widowpenalty=10000
\clubpenalty=10000
\hyphenpenalty=10000
\exhyphenpenalty=10000
\tolerance=1000
\sloppy % bu komut sayfadan taşan metni sayfa sınırlarına oturtur.

\setstretch{1.43}

%% Şekiller listesinde Şekil yazısı çıkması için gerekli satırlar
\renewcommand{\cftfigpresnum}{Şekil }
\renewcommand{\cftfigaftersnum}{.\,}% Şekil numarasından sonra nokta koyar
\setlength{\cftfignumwidth}{5.1em}
\setstretch{1.43}

%% Tablolar listesinde Tablo yazısı çıkması için gerekli satırlar
\renewcommand{\cfttabpresnum}{Tablo }
\renewcommand{\cfttabaftersnum}{.\,} % Tablo numarasından sonra nokta koyar
\setlength{\cfttabnumwidth}{5.1em}%Tablo numarası ve etiketinin genişliğini ayarlar.
\setlength{\LTpre}{-6pt}
\setlength{\LTpost}{-18pt}
\renewcommand{\cftsecaftersnum}{.}% Alt başlık numarasından sonra nokta koyar
\renewcommand{\cftsubsecaftersnum}{.}% Küçük alt başlık numarasından sonra nokta koyar

\addtocontents{toc}{~\hfill{\bf\underline{Sayfa No}}\vspace{0.3cm}\par} % İçindekiler listesinde sayfa no yazısı ekler
\addtocontents{lof}{~\hfill{\bf\underline{Sayfa No}}\vspace{0.3cm}\par} % Şekiller listesinde sayfa no yazısı ekler
\addtocontents{lot}{~\hfill{\bf\underline{Sayfa No}\vspace{0.3cm}}\par} % Çizelgeler listesinde sayfa no yazısı ekler

\arrayrulecolor{black}

\titlespacing*{\chapter}{30pt}{20pt}{20pt}
\titlespacing*{\section}{30pt}{20pt}{20pt}
\titlespacing*{\subsection}{30pt}{20pt}{20pt}
\titlespacing*{\subsubsection}{30pt}{20pt}{20pt}
\titlespacing*{\paragraph}{30pt}{20pt}{20pt}

\captionsetup[figure]{format=hang, margin={0.2cm,0cm}}
\captionsetup[table]{format=hang} %% table title align ayarı

\setcounter{secnumdepth}{5}
\setcounter{tocdepth}{5}


\setlength\cftparskip{4pt}
\setlength\cftbeforesecskip{4pt}
\setlength\cftaftertoctitleskip{4pt}
\setlength\cftbeforefigskip{4pt}
\setlength\cftbeforetabskip{4pt}

\setlength{\parskip}{2pt plus4mm minus3mm}
\setlength{\parindent}{1cm}

\captionsetup[table]{aboveskip=6pt}
\captionsetup[table]{belowskip=6pt}
\captionsetup[figure]{aboveskip=2pt}
\captionsetup[figure]{belowskip=-6pt}