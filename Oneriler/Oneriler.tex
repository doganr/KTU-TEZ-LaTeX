\chapter{ÖNERİLER VE TARTIŞMA \label{sec:Oneriler}}

Yapılan tez çalışmasında kabadan inceye yöntemi ile pankreas ilgi bölgesinin belirlenmesi ve bu ilgi bölgesinde pankreas ve pankreas tümörü segmentasyonunun gerçekleştirilmesine yönelik yaklaşımlar önerilmektedir. Önerilen yaklaşımların performans değerlendirmeleri için abdominal kontrastlı pankreas BT dilimleri kullanılmaktadır.  İki aşamalı yaklaşım otomatik  pankreas  ve pankreas tümör segmentasyonu  için  tasarlanmış olsa  da, pankreas organında görüldüğü gibi farklı yapılara sahip diğer organları segmente etmek için kolayca uyarlanabilmektedir. Önerilen yaklaşımların farklı organların segmentasyonu için uygulanması çalışmanın özgünlüğünü daha da artıracaktır. 

Önerilen yaklaşımımızın ilk aşaması (Pankreas İlgi Bölgesinin Belirlenmesi), Mask R-CNN Modeli ile İlgi Bölgesinin Belirlenmesi ve Kaba Pankreas Bölgesinin Çıkarılması olmak üzere iki ana adımdan oluşmaktadır. Yapılan literatür araştırmasına göre orijinal Mask R-CNN modelinin Çekirdek Ağı olarak ResNet 50 ve ResNet 101 kullandığı fark edilmektedir. Çekirdek Ağı’nda hangisinin uygun olduğunu belirlemek için tez çalışmasında hem ResNet 50 hem de ResNet 101 özellik çıkarıcıları test edilmektedir. Elde edilen sonuçlara göre ResNet 101, otomatik pankreas ve pankreas tümörü segmentasyonu için daha iyi performans vermektedir. Tez çalışmasının özgünlüğü artırmak ve daha etkin sonuçlar elde etmek için ilk aşamada (Pankreas İlgi Bölgesinin Belirlenmesi) EfficientNet veya Densenet gibi diğer gelişmiş modeller kullanılabilmektedir.

Önerilen yaklaşımımızın ikinci aşamasında (Pankreas ve Pankreas Tümörü Segmentasyonu), kaba pankreas bölgesi üzerinde 3B FCN Oto Kodlayıcı, 3B U-Net ve 3B U-Net++ modelleri kullanılarak segmentasyon işlemleri gerçekleştirilmektedir. 2B CNN tabanlı modellerin BT'nin üçüncü boyut bilgilerini yakalamada kısıtlamalara sahip olduğu literatür çalışmalarında bildirilmektedir. Bu nedenle bu çalışmada 3B U-Net modelleri geliştirilerek daha doğru pankreas segmentasyon sonuçları elde edilmektedir. Çalışmamızın ikinci aşamasında (Pankreas ve Pankreas Tümörü Segmentasyonu) daha başarılı sonuçların elde edilmesi yerine bu modeller yerine farklı modeller de geliştirilebilmektedir. 

Tez çalışmasında otomatik pankreas ve pankreas tümörü segmentasyonu sürecinde daha düşük boyutlu 2B alt BT görüntülerinin işlenmesi nedeniyle, GPU kapasitesi daha düşük olan çalışmamız, daha güçlü GPU kapasiteli diğer literatür çalışmalarına göre daha yüksek doğrulukta sonuçlar üretebilmektedir. Çalışmada kullanılan GPU kapasitesi daha da artırılarak elde edilen sonuçlar daha geliştirilebilmektedir. 

Tez çalışmasında birinci kısım olan Pankreas Segmentasyonu için NIH veri seti, ikinci kısım olan Pankreas ve Pankreas Tümörü Segmentasyonu için MSD veri seti kullanılmaktadır. Bu veri setlerindeki kısıtlamalardan dolayı düşük başarılı segmentasyon gerçekleştirilmektedir. Gelecek çalışmalarda kendi veri setimizi oluşturarak segmentasyon başarısı daha da artırılabilmektedir. 

