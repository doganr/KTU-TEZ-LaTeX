
% % % % % % % % % % % % % % % % % % % % % % % % % % % % % % % % % % % % % %
% % Bu satırdan sonra teziniz ile ilgili girmeniz gereken bilgiler bulunmaktadır. Kıvrık parantezler arasına size uygun bilgileri girebilirsiniz.
%%**************************************************************************
%\renewcommand{\teztipibool}{0}
%\thesistype{MS. Thesis}
%\teztipi{YÜKSEK LİSANS TEZİ}
%\teztipikucuk{Yüksek Lisans Tezi}
%%**************************************************************************
\renewcommand{\teztipibool}{1}
\thesistype{PhD. Thesis}
\teztipi{DOKTORA TEZİ}
\teztipikucuk{Doktora Tezi}
%**************************************************************************
\keywords{
	\singlespacing
	Computer tomography, Pancreas segmentation, Pancreatic tumor segmentation, Deep learning, Mask R-CNN, 3B U-Net
	%Serous effusion, Cytopathology, Machine learning, Computer aided diagnosis, Stain normalization, Nuclei detection, %Nuclei segmentation, Cell classification, Deep learning, Convolutional neural networks.
} % Anahtar kelimelerin İngilizce'sini aralarına virgül koyarak yazın

\anahtarsoz{
	\singlespacing
	Bilgisayarlı tomografi, Pankreas segmentasyonu, Pankreas tümörü segmentasyonu, Derin öğrenme, Mask R-CNN, 3B U-Net
}% Anahtar kelimelerin Türkçe'sini aralarına virgül koyarak yazın

\title{INVESTIGATION OF DIFFERENT DEEP LEARNING TECHNIQUES IN PANCREAS CANCER TISSUES SEGMENTATION} % Tez başlığınızın İngilizce'sini BÜYÜK harflerle yazın
\titlesmall{Investigation of Different Deep Learning Techniques in Pancreas Cancer Tissues Segmentation}% Tez başlığınızın İngilizce'sini küçük harflerle yazın
\baslik{PANKREAS KANSER DOKULARININ SEGMENTASYONUNDA FARKLI DERİN ÖĞRENME TEKNİKLERİNİN PERFORMANSLARININ İNCELENMESİ} % Tez başlığınızın Türkçe'sini BÜYÜK harflerle yazın
\baslikkucuk{Pankreas Kanser Dokularının Bölütlenmesinde Farklı Derin Öğrenme Tekniklerinin Performanslarının İncelenmesi}% Tez başlığınızın Türkçe'sini küçük harflerle yazın
\yazar{RAMAZAN ÖZGÜR DOĞAN}    % Tez yazarının adi ve soyadı BÜYÜK harflerle yazilmali
\yazarkucuk{Ramazan Özgür DOĞAN}    % Tez yazarının adı normal soyadı BÜYÜK harflerle yazilmali
\yunvan{Bil. Yük. Müh.}
\yorcid{0000-0001-6415-5755}
% % % % % % % % % % % % % % % % % % % % % % % % % % % % % % % % % % % % % % % % % % % % % % % % % % % % % % % % % % % % %
% % % % % % % % % % % % % % % % % % % % % % % % % % % % % % % % % % % % % % % % % % % % % % % % % % % % % % % % % % % % %
% %Aşağıdaki üniversite adı ve enstitü adında genel bir değişiklik yapmanıza gerek yoktur ama ihtiyaça halınde nasıl değişiklik yapabileceğiniz her satırn yanında yazmaktadır.
\universite{KARADENİZ TEKNİK ÜNİVERSİTESİ} %Üniveritenin adının Türkçe'sini BÜYÜK harflerle yazın
\universitekucuk{Karadeniz Teknik Üniversitesi}%Üniveritenin adının Türkçe'sini yazın
\university{KARADENIZ TECHNICAL UNIVERSITY}%Üniveritenin adının İngilizce'sini BÜYÜK harflerle yazın
\universitysmall{Karadeniz Technical University}%Üniveritenin adının İngilizce'sini yazın
\misafiruniversite{disaridan gelen hocanın üniversitesi} %Dışarıdan gelen hocanın bağlı bulunduğu üniversiteyi buraya yazın, eğer başka üniversiteden gelen hoca yoksa buraya da kendi üniversitenizin adını yazın.

\enstitu{FEN BİLİMLERİ ENSTİTÜSÜ} %Enstitünün adının Türkçe'sini BÜYÜK harflerle yazın
\enstitukucuk{Fen Bilimleri Enstitüsü} %Enstitünün adının Türkçe'sini yazın
\institute{THE GRADUATE SCHOOL OF NATURAL AND APPLIED SCIENCES}%Enstitünün adının İngilizce'sini BÜYÜK harflerle yazın
\institutesmall{The Graduate School of Natural and Applied Sciences}%Enstitünün adının İngilizce'sini yazın
\ounvan{DOKTOR (BİLGİSAYAR MÜHENDİSLİĞİ)}% alacağı ünvan

% % % % % % % % % % % % % % % % % % % % % % % % % % % % % % % % % % % % % % % % % % % % % % % % % % % % % % % % % % % % %
% % % % % % % % % % % % % % % % % % % % % % % % % % % % % % % % % % % % % % % % % % % % % % % % % % % % % % % % % % % % %

\bolum{BİLGİSAYAR MÜHENDİSLİĞİ} % Bölümünüzün adının Türkçe'sini BÜYÜK harflerle yazın
\bolumkucuk{Bilgisayar Mühendisliği} % Bölümünüzün adının Türkçe'sini yazın
\misafirbolum{Bilgisayar Mühendisliği} %Farklı anabilin dalından gelen hoca için burayı değiştirebilirsiniz.
\dept{COMPUTER ENGINEERING GRADUATE PROGRAM}% Bölümünüzün adının İngilizce'sini BÜYÜK harflerle yazın
\deptsmall{Computer Engineering Graduate Program}% Bölümünüzün adının İngilizce'sini  yazın

\supervisor{Prof. Dr. Temel KAYIKÇIOĞLU} %Danışmanınızın ünvanını İngilizce olarak yazın
\danisman{Prof. Dr. Temel KAYIKÇIOĞLU}%Danışmanınızın ünvanını Türkçe olarak yazın
\dorcid{0000-0002-6787-2415}
\juriuyesi{Prof. Dr. Onur OSMAN} %Diğer jüri üyesini buraya yazınız.
\secondreader{Prof. Dr. Temel KAYIKÇIOĞLU} %Diğer jüri üyesini buraya yazınız.
\thirdreader{Prof. Dr. Cemal KÖSE} %Diğer jüri üyesini buraya yazınız.
\forthreader{Prof. Dr. Ali GANGAL} %Diğer jüri üyesini buraya yazınız.
\fifthreader{Prof. Dr. Alper BAŞTÜRK} %Diğer jüri üyesini buraya yazınız.
\enstitumuduru{Prof. Dr. Asim KADIOĞLU} %Enstitü müdürünün adını yazınız.


\copyrightyear{2021}
\submitdate{xxxxx}  % Tezin verilme tarihi İngilizce BÜYÜK harflerle yazilmali
\submitdatesmall{xxxxx} % Tezin verilme tarihi İngilizce harflerle yazilmali
\tarih{ARALIK 2021}   % Tezin verilme tarihi Türkçe BÜYÜK harflerle yazilmali
\tarihkucuk{Aralık/2021}     % Tezin verilme tarihi Türkçe harflerle yazilmali
\onaytarihi{02/12/2021} % Tezinizin onay tarihini buraya yazınız.
\onaysayisi{1929} % Tezinizin onay sayısını buraya yazınız.
\evertarihi{23/11/2021} % Tezinizin enstitüye verildiği tarihi bura yazınız.
\evertarihii{23/11/2021} % Tezinizin enstitüye verildiği tarihi bura yazınız.
\savunmatarihi{16/12/2021} % Tezinizin savunma tarihini bura yazınız.
\onaysekli{oy birliği} % Eğer oy çokluğu ile teziniz kabul edildi ise "oy birliği" yerine "oy çokluğu" yazınız.


% % % % % % % % % % % % % % % % % % % % % % % % % % % % % % % % % % % % % % % % % % % % % % % % % % % % % % % % % % % % %
% % % % % % % % % % % % % % % % % % % % % % % % % % % % % % % % % % % % % % % % % % % % % % % % % % % % % % % % % % % % %
