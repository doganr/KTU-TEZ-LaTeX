\newpage
\pagestyle{plain}
\addtocontents{toc}{\protect\contentsline{}{\protect\numberline{}\vspace{-0.4cm}\hspace{-0.35cm}ÖZGEÇMİŞ}{}{}}
\begin{center}
{\bf ÖZGEÇMİŞ}
\end{center}
29.04.1987 tarihinde Trabzon’da doğdu. İlkokul, ortaokul ve lise öğrenimini Trabzon’da tamamladı. 2006 yılında Karadeniz Teknik Üniversitesi Mühendislik Fakültesi Bilgisayar Mühendisliği Bölümü’nde lisans programına başladı. 2011 yılında Bilgisayar Mühendisliği Bölümü’nden Bilgisayar Mühendisi unvanı ile mezun oldu. 2011-2012 bahar döneminde Bilgisayar Mühendisliği Anabilim Dalı’nda yüksek lisans eğitimine başladı ve aynı yıl  Temmuz ayında Gümüşhane Üniversitesi, Gümüşhane Meslek Yüksekokulu Bilgisayar Teknolojileri bölümüne Öğretim Görevlisi ünvanı ile göreve başladı. 2015 yılında yüksek lisans eğitimini tamamlayarak, 2016 yılında Karadeniz Teknik Üniversitesi Fen Bilimleri Enstitüsü Bilgisayar Mühendisliği Anabilim Dalı’nda doktora eğitimine başladı. 2019-2020 yılları arasında Amerika Birleşik Devletleri, OHIO eyaleti, Youngstown State Üniversitesinde davetli araştırmacı olarak çalıştı. 2021 yılı itibari ile Gümüşhane Üniversitesi, Mühendislik ve Doğa Bilimleri Fakültesi, Yazılım Mühendisliği Bölümüne görevlendirildi ve halen bu bölümde çalışmaya devam etmektedir. 

3’ü SCI/expanded indeksli dergilerde, 1’ i diğer dergilerde, 6’si ulusal ve uluslararası konferanslarda olmak üzere toplam 10 yayını mevcuttur. Yabancı dil olarak İngilizce bilmektedir. Yayınları aşağıda verilmiştir.
\setstretch{1.43}

\singlespacing 
\singlespacing

\indent \textbf{Uluslararası hakemli dergilerde yayınlanan makaleler (SCI/SCI-E)}

\begin{enumerate}
\item[1.] Dogan, R. O., Dogan, H., Bayrak, C., \& Kayikcioglu, T. (2021). A Two-Phase Approach using Mask R-CNN and 3D U-Net for High-Accuracy Automatic Segmentation of Pancreas in CT Imaging. Computer Methods and Programs in Biomedicine, 207, 106141.
\item[2.] Otkun, O., Dogan, R. O., \& Akpinar, A. (2015). Doğrusal hareketli sürekli miknatisli senkron motorun yapay sinir ağ tabanli skaler hiz denetimi. Gazi Üniversitesi Mühendislik Mimarlık Fakültesi Dergisi, 30(3).
\item[3.] Otkun, O., Dogan, R. O., \& Akpinar, A. (2014). Performance Analysis of Linear Permanent Magnets Synchronous Motor by Scalar Control Method (v/f). In Applied Mechanics and Materials (Vol. 666, pp. 199-202). Trans Tech Publications Ltd.
\end{enumerate}


\indent \textbf{Diğer dergilerde yayınlanan makaleler}

\begin{enumerate}
\item[1.] Dogan, R. O., Kayikcioglu, T., Yagci, Y., \& Yildirim, O. Elektronik Sağlık Kayıtlarının WCF Web Servisleri Kullanılarak Aktarılması ve Depolanması. Süleyman Demirel Üniversitesi Fen Bilimleri Enstitüsü Dergisi, 22(1), 232-236.
\end{enumerate}


\indent \textbf{Hakemli konferans/sempozyumların bildiri kitaplarında yer alan yayınlar}

\begin{enumerate}
\item[1.] Dogan, R. O., \& Kayikcioglu, T, (2018, May). R-peaks detection with convolutional neural network in electrocardiogram signal. In 2018 26th Signal Processing and Communications Applications Conference (SIU) (pp. 1-4). IEEE.
\item[2.] Dogan, R. O., Kayikcioglu, T., \& Yagci, Y., (2017, May). Real time sending and monitoring electronic health records. In 2017 25th Signal Processing and Communications Applications Conference (SIU) (pp. 1-4). IEEE.
\item[3.] Yildirim, O., Dogan, R. O., Kaya, I., \& Kayikcioglu, T., (2016, October). Health-monitoring system based tele-tip for elders. In 2016 Medical Technologies National Congress (TIPTEKNO) (pp. 1-4). IEEE.
\item[4.] Dogan, R. O., \& Kayikcioglu, T. (2016, May). Remote patient monitoring and Electronic Health Record system based on web services. In 2016 24th Signal Processing and Communication Application Conference (SIU) (pp. 1785-1788). IEEE.
\item[5.] Dogan, R. O., Dogan, H., \& Kose, C., (2015, May). Virtual mouse control with hand gesture information extraction and tracking. In 2015 23nd Signal Processing and Communications Applications Conference (SIU) (pp. 1893-1896). IEEE.
\item[6.] Dogan, R. O., \& Kose, C. (2014, April). Computer monitoring and control with hand movements. In 2014 22nd Signal Processing and Communications Applications Conference (SIU) (pp. 2110-2113). IEEE.
\end{enumerate}
\thispagestyle{empty}



  





